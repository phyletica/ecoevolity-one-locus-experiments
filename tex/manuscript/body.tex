\section{Introduction}

How well can we infer shared divergence times given only a single locus?

How does a full-likelihood method that assumes unlinked characters
compare to an ABC method that correctly assumes linked sites?

\section{Methods}

We used
\ecoevolity \citep[Version 0.2.1 (commit 8ac8228);][]{Oaks2018ecoevolity}
for all full-likelihood analsyes and
\pymsbayes \citep[Version 0.2.4;][]{Oaks2014dpp}
for all ABC analyses.
We chose the settings for simulating and analyzing data with these tools to
maximize the comparability of the results.
For both methods, we used the same distributions from which parameter values
were drawn for simulating the data as priors in the subsequent analyses of the
simulated datasets.
As a result, there were no model violations except for \ecoevolity, where we
were violated the assumption that all characters are unlinked.

\subsection{Simulation-based analyses with \ecoevolity}

To simulate 500 \datasets with a single locus of length 500, 1000, 2000, and
10,000 sites,
we used the \simcoevolity tool.
For each \dataset, there were five pairs of populations with 20 copies of the
locus sampled from each population.
We assumed that all five pairs share the same rate of mutation, which was set
to 1.0, such that time and effective population sizes are scaled by the
mutation rate.
We treated the number and timing of divergence events and the assignment of the
pairs of populations to those events as a random variable under a Dirichlet
process with a concentration parameter of 2.22543 (a prior mean number of
events of three) and base distribution of $\etime \sim \dexponential{0.01}$;
the base distribution is the prior on divergence times.

The model implemented in \ecoevolity assumes biallelic characters that evolve
along gene trees according to a continutuous-time Markov chain (CTMC) model of
state substitution.
For this CTMC model, we constrained the equilibrium frequencies of the two
states to be equal.

We drew the effective size of each descendant population of each pair
from a gamma distribution,
$\epopsize[\descendantpopindex{}] \sim \dgamma{10}{0.0005}$.
The size of each ancestral population relative to the mean
size of the descendant populations was drawn from
another gamma distribution,
$\rootrelativepopsize \sim \dgamma{100}{1}$.

For each pair of populations, we assumed the locus was biparentally inherited
and diploid (i.e., we set the ploidy to 2.0).
However, this is directly translatable to a uniparentally inherited, haploid
locus (e.g., a mitochondrial locus) by simply multiplying the effective
population sizes by four.
In other words, we would get the same results if we set the ploidy to 0.5 and
changed the distribution on the descendant population size to
\dgamma{10}{0.002}.

We ran three independent MCMC chains for 30,000 generations, sampling every
20th generation for 1501 samples from each chain, including the initial state.
For each analyzed \dataset, we removed the first 401 samples from each chain
before summarizing the posterior across the remaining 3300 samples.
We confirmed this was an adequate burn-in by checking the potential scale
reduction factor \citep[PSRF; the square root of Equation 1.1 in][]{Brooks1998}
and effective sample size \citep[ESS;][]{Gong2014} of the log-likelihood and
parameters.

\subsection{Simulation-based analyses with \pymsbayes}

For all analyses with \pymsbayes, we used the \dppmsbayes model
\citep{Oaks2014dpp}.
We used the same distributions on parameters as we did for \ecoevolity, with
two exceptions.
First, \dppmsbayes parameterizes population sizes differently than \ecoevolity.
For \dppmsbayes, priors are specified for $4\epopsize{}\murate{}$, rather than
on \epopsize{} and \murate{}.
Accordingly, we used a distribution of \dgamma{10}{0.002} on
$4\epopsize{}\murate{}$ to ensure comparability with the \ecoevolity results.
Second, for \dppmsbayes, we used a Jukes-Cantor model of nucleotide
substitution \citep{JC1969}; the model we used in \ecoevolity is the two-state
equivalent.

We simulated 500,000 samples from the joint prior distribution, and used four
summary statistics calculated from the simulated locus alignment of each pair:
\begin{enumerate}
    \item The number of segregating sites \citep[$\theta_W$;][]{Watterson1975},
    \item the average number of pairwise differences across all gene copies
        \citep[$\pi$;][]{NeiLi1979},
    \item the net number of pairwise differences between the two populations
        \citep[Equation 25 in][]{NeiLi1979}, and
    \item the standard deviation in the difference between $\pi$ and $\theta_W$
        \citep{Tajima1989}.
\end{enumerate}
We used the mean and standard deviation from the prior to standardize each
statistic.
Using the same standardized statistics calculated from each simulated observed
\dataset, we then retained the 2,000 prior samples that had the smallest
Euclidean distance as the sample from the approximate posterior.

\section{Results}

In general, neither \ecoevolity nor \dppmsbayes is doing particularly well with
just one locus.
\ecoevolity is doing a bit better by most metrics
(\cref{fig:divtimes,fig:nevents,fig:leafsizes,fig:rootsizes}),
which suggests we are better off violating the assumption of unlinked sites
rather than resorting to ABC.

\ifembed{
\embedWidthFigure{1.0}{../../simulations/validation/plots/div-time-scatter.pdf}{
    Caption \ldots
}{fig:divtimes}

}{}

\ifembed{
\embedWidthFigure{1.0}{../../simulations/validation/plots/nevents.pdf}{
    Comparing the ability of \ecoevolity (top row) and \dppmsbayes (bottom row)
    to estimate the number of divergence events given a locus of length 500,
    1000, 2000, and 10,000 sites.
    Each plot shows the results of the analyses of 500 simulated \datasets,
    each with five population pairs;
    the number of \datasets that fall within each possible cell
    of true versus estimated numbers of events is shown, and cells with
    more \datasets are shaded darker.
    \neventplotannotations
    \weusedmatplotlib
}{fig:nevents}

}{}

\ifembed{
\embedWidthFigure{1.0}{../../simulations/validation/plots/leaf-pop-size-scatter.pdf}{
    Comparing the ability of \ecoevolity (top row) and \dppmsbayes (bottom row)
    to estimate the effective sizes of the descendant populations (scaled by
    the mutation rate) given a locus of length 500, 1000, 2000, and 10,000
    sites.
    Each plotted circle and associated error bars represent the posterior mean
    and 95\% credible interval for the time that a pair of populations
    diverged.
    Each plot consists of 5000 estimates---500 simulated \datasets, each with
    five pairs of populations.
    \accuracyscatterplotannotations{\epopsize{}\murate{}}
    \weusedmatplotlib
}{fig:leafsizes}

}{}

\ifembed{
\embedWidthFigure{1.0}{../../simulations/validation/plots/root-pop-size-scatter.pdf}{
    Comparing the ability of \ecoevolity (top row) and \dppmsbayes (bottom row)
    to estimate the effective sizes of the ancestral populations (scaled by the
    mutation rate) given a locus of length 500, 1000, 2000, and 10,000 sites.
    Each plotted circle and associated error bars represent the posterior mean
    and 95\% credible interval for the time that a pair of populations
    diverged.
    Each plot consists of 2500 estimates---500 simulated \datasets, each with
    five pairs of populations.
    \accuracyscatterplotannotations{\epopsize{}\murate{}}
    \weusedmatplotlib
}{fig:rootsizes}

}{}


The PSRF
(Fig.\
S\ref{fig:psrflikelihood}
\&
S\ref{fig:psrfdivtimes})
and ESS
(Fig.\
S\ref{fig:esslikelihood},
S\ref{fig:essdivtimes},
\&
S\ref{fig:essrootsizes})
values for the log likelihood score and parameters
indicate that MCMC chains run with \ecoevolity converged and mixed well.

\section{Discussion}

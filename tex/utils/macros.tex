\newcommand{\citationNeeded}{\textcolor{magenta}{\textbf{[CITATION NEEDED!]}}\xspace}
\newcommand{\tableNeeded}{\textcolor{magenta}{\textbf{[TABLE NEEDED!]}}\xspace}
\newcommand{\figureNeeded}{\textcolor{magenta}{\textbf{[FIGURE NEEDED!]}}\xspace}
\newcommand{\highLight}[1]{\textcolor{magenta}{\MakeUppercase{#1}}}

\newcommand{\datasets}{data sets\xspace}
\newcommand{\dataset}{data set\xspace}

\newcommand{\editorialNote}[1]{\textcolor{red}{[\textit{#1}]}}
\newcommand{\ignore}[1]{}
\newcommand{\addTail}[1]{\textit{#1}.---}
\newcommand{\super}[1]{\ensuremath{^{\textrm{#1}}}}
\newcommand{\sub}[1]{\ensuremath{_{\textrm{#1}}}}
\newcommand{\dC}{\ensuremath{^\circ{\textrm{C}}}}
\newcommand{\tb}{\hspace{2em}}
\newcommand{\tn}{\tabularnewline}
\newcommand{\spp}[1]{\textit{#1}}

\providecommand{\e}[1]{\ensuremath{\times 10^{#1}}}

\newcommand{\change}[2]{{\color{red} #2}\xspace}
\newcommand{\thought}[1]{\textcolor{purple}{THOUGHT: #1}}

\newcommand{\widthFigure}[5]{\begin{figure}[htbp]
\begin{center}
    \includegraphics[width=#1\textwidth]{#2}
    \captionsetup{#3}
    \caption{#4}
    \label{#5}
    \end{center}
    \end{figure}}

\newcommand{\heightFigure}[5]{\begin{figure}[htbp]
\begin{center}
    \includegraphics[height=#1\textheight]{#2}
    \captionsetup{#3}
    \caption{#4}
    \label{#5}
    \end{center}
    \end{figure}}

\newcommand{\smartFigure}[4]{%
    \begin{figure}[htbp]
        \begin{center}
            \includegraphics[width=\textwidth,height=0.95\textheight,keepaspectratio]{#1}
            \captionsetup{#2}
            \caption{#3}
            \label{#4}
        \end{center}
    \end{figure}
}

\newcommand{\mFigure}[3]{\smartFigure{#1}{listformat=figList}{#2}{#3}\clearpage}
\newcommand{\embedHeightFigure}[4]{\heightFigure{#1}{#2}{listformat=figList}{#3}{#4}}
\newcommand{\embedWidthFigure}[4]{\widthFigure{#1}{#2}{listformat=figList}{#3}{#4}}
\newcommand{\siFigure}[3]{\smartFigure{#1}{name=Figure S, labelformat=noSpace, listformat=sFigList}{#2}{#3}\clearpage}

\newcommand{\weusedmatplotlib}{We generated the plot using matplotlib Version
    2.0.0 \citep{matplotlib}.}
\newcommand{\weusedggplot}{We generated the plots with ggplot2 Version 2.2.1
    \citep{ggplot2}.}
\newcommand{\weusedggridges}{We generated the plots with ggridges Version 0.4.1
    \citep{ggridges041} and ggplot2 Version 2.2.1 \citep{ggplot2}.}
\newcommand{\neventplotannotations}{For each plot,
    the proportion of \datasets for which the number of events with the largest
    posterior probability matched the true number of events---$p(\hat{\nevents}
    = \nevents)$---is shown in the upper left corner,
    the median posterior probability of the correct number of events across all
    \datasets---$\widetilde{p(\nevents|\alldata)}$---is shown in the upper
    right corner, and
    the proportion of \datasets for which the true divergence model was
    included in the 95\% credible set---$p(\nevents \in
    \textrm{CS})$---is shown in the lower right.
}
\newcommand{\accuracyscatterplotannotations}[1]{For each plot, the
    root-mean-square error (RMSE) and the proportion of estimates for which the
    95\% credible interval contained the true value---$p(#1 \in
    \textrm{CI})$---is given.
}
\newcommand{\neventsshadingdescription}{The number of \datasets that fall
    within each possible cell of true versus estimated numbers of events is
    shown, and cells with more \datasets are shaded darker.
}

%% macro to make long strings breakable over lines
\makeatletter
\def\breakable#1{\xHyphen@te#1$\unskip}
\def\xHyphen@te{\@ifnextchar${\@gobble}{\sw@p{\allowbreak{}\xHyphen@te}}}
% \def\xHyphen@te{\@ifnextchar${\@gobble}{\sw@p{\hskip 0pt plus 1pt\xHyphen@te}}}
\def\sw@p#1#2{#2#1}
\makeatother
